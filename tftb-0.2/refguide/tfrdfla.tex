% This is part of the TFTB Reference Manual.
% Copyright (C) 1996 CNRS (France) and Rice University (US).
% See the file refguide.tex for copying conditions.


\renewcommand{\footnoterule}{}
\markright{tfrdfla}
\section*{\hspace*{-1.6cm} tfrdfla}

\vspace*{-.4cm}
\hspace*{-1.6cm}\rule[0in]{16.5cm}{.02cm}
\vspace*{.2cm}

{\bf \large \sf Purpose}\\
\hspace*{1.5cm}
\begin{minipage}[t]{13.5cm}
D-Flandrin time-frequency distribution.
\end{minipage}
\vspace*{.5cm}

{\bf \large \sf Synopsis}\\
\hspace*{1.5cm}
\begin{minipage}[t]{13.5cm}
\begin{verbatim}
[tfr,t,f] = tfrdfla(x)
[tfr,t,f] = tfrdfla(x,t)
[tfr,t,f] = tfrdfla(x,t,fmin,fmax)
[tfr,t,f] = tfrdfla(x,t,fmin,fmax,N)
[tfr,t,f] = tfrdfla(x,t,fmin,fmax,N,trace)
\end{verbatim}
\end{minipage}
\vspace*{.5cm}

{\bf \large \sf Description}\\
\hspace*{1.5cm}
\begin{minipage}[t]{13.5cm}
        {\ty tfrdfla} generates the auto- or cross- D-Flandrin
        distribution.  This distribution has the
        following expression :
\begin{eqnarray*}
D_x(t,\nu)=\nu\ \int_{-\infty}^{+\infty} (1-(\gamma/4)^2)\
X\left(\nu (1-\gamma/4)^2\right)\ X^*\left(\nu (1+\gamma/4)^2\right)\
e^{-j2\pi\gamma t\nu}\ d\gamma.
\end{eqnarray*}


\hspace*{-.5cm}\begin{tabular*}{14cm}{p{1.5cm} p{8.5cm} c}
Name & Description & Default value\\
\hline
        {\ty x} & signal (in time) to be analyzed. If {\ty x=[x1 x2]}, {\ty tfrdfla} 
           computes the cross-D-Flandrin distribution ({\ty Nx=length(X)})\\
        {\ty t} & time instant(s) on which the {\ty tfr} is evaluated & {\ty (1:Nx)}\\
        {\ty fmin, fmax} & respectively lower and upper frequency bounds of 
           the analyzed signal. These parameters fix the equivalent 
           frequency bandwidth (expressed in Hz). When unspecified, you
           have to enter them at the command line from the plot of the
           spectrum. {\ty fmin} and {\ty fmax} must be $>0$ and $\leq 0.5$\\         
        {\ty N} & number of analyzed voices & auto\footnote{This value,
	determined from {\ty fmin} and {\ty fmax}, is the 
	next-power-of-two of the minimum value checking the non-overlapping
	condition in the fast Mellin transform.}\\
        {\ty trace} & if nonzero, the progression of the algorithm is shown
                                           & {\ty 0}\\

\hline\end{tabular*}\end{minipage} \newpage
\hspace*{1.5cm}\begin{minipage}[t]{13.5cm}
\hspace*{-.5cm}\begin{tabular*}{14cm}{p{1.5cm} p{8.5cm} c}
Name & Description & Default value\\\hline

     \hline {\ty tfr} & time-frequency matrix containing the coefficients of the 
           decomposition (abscissa correspond to uniformly sampled
           time, and ordonates correspond to a geometrically sampled 
           frequency). First row of {\ty tfr} corresponds to the lowest 
           frequency\\
        {\ty f} & vector of normalized frequencies (geometrically sampled 
           from {\ty fmin} to {\ty fmax})\\

\hline
\end{tabular*}
\vspace*{.2cm}

When called without output arguments, {\ty tfrdfla} runs {\ty tfrqview}.
\end{minipage}
\vspace*{.5cm}


{\bf \large \sf Example}
\begin{verbatim}
         sig=altes(64,0.1,0.45); 
         tfrdfla(sig); 
\end{verbatim}
\vspace*{.5cm}

{\bf \large \sf See Also}\\
\hspace*{1.5cm}
\begin{minipage}[t]{13.5cm}
all the {\ty tfr*} functions.
\end{minipage}
\vspace*{.5cm}


{\bf \large \sf Reference}\\
\hspace*{1.5cm}
\begin{minipage}[t]{13.5cm}
[1] P. Flandrin ``Temps-fr�quence'' Trait� des Nouvelles Technologies,
		  s�rie Traitement du Signal, Herm\`es, 1993.
\end{minipage}
