% This is part of the TFTB Reference Manual.
% Copyright (C) 1996 CNRS (France) and Rice University (US).
% See the file refguide.tex for copying conditions.


\markright{holder}
\section*{\hspace*{-1.6cm} holder}

\vspace*{-.4cm}
\hspace*{-1.6cm}\rule[0in]{16.5cm}{.02cm}
\vspace*{.2cm}



{\bf \large \sf Purpose}\\
\hspace*{1.5cm}
\begin{minipage}[t]{13.5cm}
H�lder exponent estimation through an affine TFR.
\end{minipage}
\vspace*{.2cm}


{\bf \large \sf Synopsis}\\
\hspace*{1.5cm}
\begin{minipage}[t]{13.5cm}
\begin{verbatim}
h = holder(tfr,f)
h = holder(tfr,f,n1)
h = holder(tfr,f,n1,n2)
h = holder(tfr,f,n1,n2,t)
\end{verbatim}
\end{minipage}
\vspace*{.5cm}


{\bf \large \sf Description}\\
\hspace*{1.5cm}
\begin{minipage}[t]{13.5cm}
        {\ty holder} estimates the H�lder exponent of a signal through an
        affine time-frequency representation of it.  \\

\hspace*{-.5cm}\begin{tabular*}{14cm}{p{1.5cm} p{8.5cm} c}
Name & Description & Default value\\
\hline
        {\ty tfr} & affine time-frequency representation\\ 
        {\ty f}   & frequency values of the spectral analysis\\ 
        {\ty n1}  & indice of the  minimum frequency for the linear regression  
                                          & {\ty 1}\\
        {\ty n2}  & indice of the  maximum frequency for the linear regression  
                                          & {\ty length(f)}\\
        {\ty t} & time vector. If {\ty t} is omitted, the function returns the
            global estimate of the H�lder exponent. Otherwise, it
            returns the local estimates {\ty h(t)} at the instants specified
            in {\ty t}\\
 \hline {\ty h} & output value (if {\ty t} omitted) or vector (otherwise) containing
            the H�lder estimate(s)\\

\hline
\end{tabular*}
\end{minipage}
\vspace*{.5cm}


{\bf \large \sf Example}\\
\hspace*{1.5cm}
\begin{minipage}[t]{13.5cm}
For instance, we consider a 64-points Lipschitz singularity (see {\ty
  anasing}) of strength {\ty h=0}, centered at {\ty t0=32}, analyze it with the
  scalogram (Morlet wavelet with half-length = 4), and estimate its H�lder
  exponent,
\begin{verbatim}
         sig=anasing(64);
         [tfr,t,f]=tfrscalo(sig,1:64,4,0.01,0.5,256,1);
         h=holder(tfr,f,1,256,1:64);
\end{verbatim}
the value obtained at time {\ty t0} is a good estimation of {\ty h} (we obtain
{\ty h(t0)=-0.0381}).\\ 
\end{minipage}

%\newpage

{\bf \large \sf See Also}\\
\hspace*{1.5cm}
\begin{minipage}[t]{13.5cm}
\begin{verbatim}
anastep, anapulse, anabpsk, doppler.
\end{verbatim}
\end{minipage}
\vspace*{.5cm}


{\bf \large \sf Reference}\\
\hspace*{1.5cm}
\begin{minipage}[t]{13.5cm}
[1] S. Jaffard ``Exposants de H�lder en des points donn�s et coefficients
d'ondelettes'' C.R. de l'Acad�mie des Sciences, Paris, t. 308, S�rie I,
p. 79-81, 1989.\\

[2] P. Gon�alv�s, P. Flandrin ``Scaling Exponents Estimation From
Time-Scale Energy Distributions'' IEEE ICASSP-92, pp. V.157 - V.160, San
Fransisco 1992.
\end{minipage}

