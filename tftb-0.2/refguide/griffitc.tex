% This is part of the TFTB Reference Manual.
% Copyright (C) 1996 CNRS (France) and Rice University (US).
% See the file refguide.tex for copying conditions.



\markright{griffitc}
\hspace*{-1.6cm}{\Large \bf griffitc}

\vspace*{-.4cm}
\hspace*{-1.6cm}\rule[0in]{16.5cm}{.02cm}
\vspace*{.2cm}



{\bf \large \sf Purpose}\\
\hspace*{1.5cm}
\begin{minipage}[t]{13.5cm}
Test signal example C of Griffiths' paper. 
\end{minipage}
\vspace*{.5cm}


{\bf \large \sf Synopsis}\\
\hspace*{1.5cm}
\begin{minipage}[t]{13.5cm}
\begin{verbatim}
[sig,iflaws] = griffitc
[sig,iflaws] = griffitc(N)
[sig,iflaws] = griffitc(N,SNR)
\end{verbatim}
\end{minipage}
\vspace*{.5cm}


{\bf \large \sf Description}\\
\hspace*{1.5cm}
\begin{minipage}[t]{13.5cm}
        {\ty griffitc} generates the test signal of the example C of the
        paper of Griffiths [1].\\

\hspace*{-.5cm}\begin{tabular*}{14cm}{p{1.5cm} p{8.5cm} c}
Name & Description & Default value\\
\hline
        {\ty N}      & signal length         & {\ty 200}\\
        {\ty SNR}    & signal to noise ratio & {\ty 25 dB}\\
        {\ty sig}    & output signal\\
        {\ty iflaws} & instantaneous frequency laws of the 3 components\\

\hline
\end{tabular*}

\end{minipage}
\vspace*{1cm}


{\bf \large \sf Example}
\begin{verbatim}
         [sig,iflaws]=griffitc; 
         plotifl(1:200,iflaws); grid;
\end{verbatim}
\vspace*{.5cm}


{\bf \large \sf Reference}\\
\hspace*{1.5cm}
\begin{minipage}[t]{13.5cm}
[1] ??
\end{minipage}



